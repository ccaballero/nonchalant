\documentclass[letter,12pt]{article}

\usepackage[T1]{fontenc}
\usepackage{lmodern}
\usepackage{textcomp}
\renewcommand*\familydefault{\sfdefault}

\usepackage[spanish]{babel}
\usepackage[utf8x]{inputenc}

\usepackage[pdftex]{graphicx}
\usepackage{pifont}
\usepackage[
pdfauthor={Carlos Caballero Burgoa},%
pdftitle={Proyecto nonchalant},%
colorlinks,%
citecolor=black,%
filecolor=black,%
linkcolor=black,%
%urlcolor=black
pdftex]{hyperref}

\usepackage{fancyhdr}
\usepackage{lastpage}
\pagestyle{fancy}

% Para la primera página
\fancypagestyle{plain}{
\fancyhead[l]{}
\fancyhead[r]{}
\fancyhead[c]{}
\renewcommand{\headrulewidth}{0.5pt}
\fancyfoot[l]{SCESI \\ Sociedad Científica de Estudiantes de Sistemas e Informática}
\fancyfoot[c]{}
\fancyfoot[r]{\thepage/\pageref{LastPage}}
\renewcommand{\footrulewidth}{0.5pt}}

% Para el resto de páginas
\lhead{Proyecto nonchalant}
\chead{}
\rhead{\includegraphics[width=0.1\textwidth]{scesi.png}}
\renewcommand{\headrulewidth}{0.4pt}
\lfoot{SCESI \\ Sociedad Científica de Estudiantes de Sistemas e Informática \\ 
\url {http://scesi.fcyt.umss.edu.bo}}
\cfoot{}
\rfoot{\thepage/\pageref{LastPage}}
\renewcommand{\footrulewidth}{0.4pt}

\title{\bf nonchalant}
\author{Carlos Eduardo Caballero Burgoa} 

\begin{document}
\maketitle
\begin{center}\includegraphics[width=0.48\textwidth]{nonchalant.png}\end{center}
\begin{center}\url {http://scesi.fcyt.umss.edu.bo}\end{center}
\pagebreak

\tableofcontents
\pagebreak

\section{Introducción}


\section{Antecedentes}


\section{Definición del Problema}


https://c9.io/

\section{Objetivo General}
Reescribir el conjunto funciones básicas de un sistema operativo utilizando el lenguaje PHP en su version (5.4), de tal forma que pueda estructurarse una terminal de comandos para asi poder entender de manera eficaz tanto el lenguaje PHP como el lenguage Shell.

\section{Objetivos Específicos}
\begin{itemize}
\item Implementar un cargador de clases para objetos en PHP.
\item Implementar un parser para evaluar expresiones de entrada.
\item Implementar un parser para la interpretación de parametros y opciones.
\item Implementar las funciones base del kernel de un sistema operativo.
\item Implementar los comandos necesarios para el manejo de un sistema de ficheros.
\item Implementar las primitivas necesarias para la construcción de un sistema de ficheros.
\item Implementar de un entorno de escritorio para el manejo del proyecto.
\end{itemize}

\section{Herramientas}

\section{Justificación}

\begin{thebibliography}{99}
\end{thebibliography}

\end{document}
