\documentclass[letterpaper,11pt]{article}

\usepackage[T1]{fontenc}
\usepackage{lmodern}
\usepackage{textcomp}

\usepackage[spanish]{babel}
\usepackage[utf8x]{inputenc}

\usepackage[pdftex]{graphicx}
\usepackage{pifont}

\usepackage[
pdfauthor={Carlos Caballero Burgoa},%
pdftitle={Proyecto nonchalant},%
colorlinks,%
citecolor=black,%
filecolor=black,%
linkcolor=black,%
%urlcolor=black
pdftex]{hyperref}

\usepackage{fancyhdr}
\usepackage{lastpage}
\pagestyle{fancy}

% Para la primera página
\fancypagestyle{plain}{
\fancyhead[l]{}
\fancyhead[r]{}
\fancyhead[c]{}
\renewcommand{\headrulewidth}{0.5pt}
\fancyfoot[l]{SCESI \\ Sociedad Científica de Estudiantes de Sistemas e
Informática}
\fancyfoot[c]{}
\fancyfoot[r]{\thepage/\pageref{LastPage}}
\renewcommand{\footrulewidth}{0.5pt}}

% Para el resto de páginas
\lhead{Proyecto nonchalant}
\chead{}
\rhead{\includegraphics[width=0.1\textwidth]{scesi.png}}
\renewcommand{\headrulewidth}{0.4pt}
\lfoot{SCESI \\ Sociedad Científica de Estudiantes de Sistemas e Informática\\
\url {http://scesi.fcyt.umss.edu.bo}}
\cfoot{}
\rfoot{\thepage/\pageref{LastPage}}
\renewcommand{\footrulewidth}{0.4pt}

\title{\bf nonchalant}
\author{
    Carlos Eduardo Caballero Burgoa (jacobian) \\
    Juan Pablo Mejía Veizaga (jmejia) \\
    Vladimir Céspedes López (vladwelt) \\
    José Valdivia Ignacio (pepex system V) \\
    Franz López Choque (aquafranx) \\
    Elvis Rogelio Ramírez Iriarte (elvikito) \\
}

\begin{document}
\maketitle
\begin{center}\includegraphics[width=0.48\textwidth]
{nonchalant.png}\end{center}
\begin{center}\url {http://scesi.fcyt.umss.edu.bo}\end{center}
\pagebreak

\tableofcontents
\pagebreak

\section{Introducción}
Desde los primeros años de la computación existió el problema de interacción
entre el usuario y los programas, si bien se crearon una multitud de
mecanismos para solucionar estos inconvenientes, uno de estos métodos ha
perdurado hasta nuestros días, esa es la linea de comandos.

La contraparte de la linea de comandos (CLI) es la interfaz gráfica de usuario
(GUI) que ofrece una estética mejorada y una mayor simplificación, a costa de
un mayor consumo de recursos computacionales, y, en general, de una reducción
de la funcionalidad alcanzable. Asimismo aparece el problema de una mayor
vulnerabilidad por complejidad.\footnote{Línea de comandos:
https://es.wikipedia.org/wiki/Línea\_de\_comandos.}

En este documento se plantea la construcción de una linea de comandos que
pueda ser desplegada como una aplicación web. A partir de esta forma de
interfaz, construir las funcionalidades necesarias del lado del servidor para
tener un pequeño conjunto de las tareas cotidianas del kernel de un sistema
operativo.

\section{Antecedentes}
Hace años que los fabricantes de sistemas operativos —como Microsoft o Apple—
dedican ingentes recursos a ocultar cómo funcionan realmente los ordenadores,
se supone que con la idea de simplificar su uso. Para ello, algunos de sus
mejores ingenieros han inventado toda clase de metáforas visuales e interfaces
gráficas, lo cual ha permitido que mucha gente se acerque a los ordenadores
personales sin sentir pánico o sin provocar grandes gastos de formación de
personal a sus empresas. Pero, lamentablemente, construir ese muro de
metáforas en forma de interfaz gráfica entre el ordenador y el usuario
(conocida como GUI) ha tenido un coste social y cultural muy notable, al
contribuir decisivamente a que la tecnología que subyace al ordenador se
perciba como algo mágico, sin conexión alguna entre causas y efectos,
recubriendo de un formidable manto de ignorancia todo lo que realmente
sucede.\cite{Stephenson}

Yéndonos mas hacia el terreno de lo local, una de las mayores brechas
observadas es el escaso acceso a medio físicos computacionales, para el
aprendizaje de sistemas operativos, y mas propiamente para el acceso a una
linea de comandos tipo UNIX, esto conlleva a una insuficiente formación acerca
de las funciones disponibles en un sistema operativo.

Aun entre aquellos cuyo destino pretendido es el estudio de las ciencias de la
computación, existe el problema de la complejidad del lenguaje nativo
(Lenguaje C) y las arquitecturas de un sistema operativo moderno, lo que en
gran medida dificulta el aprendizaje de las instrucciones básicas de la linea
de comandos y de su funcionamiento a aquellos que aún están comenzando sus
estudios.

Es notorio en este punto que la escasa disponibilidad de medios educativos
alternativos a los tradicionales para el aprendizaje de sistemas operativos es
una de las mayores necesidades para evitar que se tenga un conocimiento muy
superficial del tema.

\section{Definición del Problema}
Como se mencionó anteriormente, las GUIs han opacado el uso de la linea de
comandos, consecuencia de esto existe una escasez de conocimiento acerca de
como desempeñarse delante de una terminal, y por ende se ha perdido cierta
parte del conocimiento mismo del funcionamiento de un sistema operativo
reduciendo las posibilidades de un eficiente aprovechamiento de las funciones
disponibles que este posee.

También se ve un falta de fomento hacia los estudiantes que ocasiona que estos
no se atrevan a usar la linea de comandos como un habito formal, siendo una
carencia preocupante, especialmente para aquellos que estudian alguna rama
afín a las ciencias de la computación, esto hace que a larga aquellos
requieran siempre de una interfaz de alto nivel para la realización de sus
tareas, es decir, se crea una dependencia de herramientas poco versátiles y
muy especializadas.

A todo esto se suma la inexistencia de una herramienta libre (algo que
cualquier usuario puede utilizar, instalar, y modificar) con las capacidades
necesarias para la extensión de las funcionalidades de un sistema operativo.

Por lo mencionado se define el problema como:

\emph{“El desconocimiento de las funciones de la linea de comandos ocasiona un
escaso interés en el uso, administración, y automatización de tareas por medio
de esta.”}

\section{Objetivo General}
Reescribir el conjunto funciones básicas de un sistema operativo, de tal forma
que pueda estructurarse una linea de comandos, para así poder facilitar el
entendimiento de las funciones que esta posee de manera eficaz.

\section{Objetivos Específicos}
\begin{itemize}
\item Crear una interface de linea de comandos que pueda ser desplegada por
medio de un cliente Web.
\item Crear un conjunto básico de llamadas al sistema (syscalls) de modo que
estas puedan enriquecer la variedad de instrucciones posibles de la linea de
comandos.
\item Crear un conjunto básico de instrucciones que provean a la aplicación de
la funcionalidad básica y cotidiana en los sistemas operativos.
\item Diseñar e implementar una arquitectura de software que permita extender
el conjunto de funcionalidades establecidas.
\item Proveer a la aplicación construida de una licencia que permita la
instalación, distribución, y modificación de su código fuente.
\end{itemize}

\section{Herramientas}
Siendo este un proyecto creado por el grupo PHP de la Sociedad Científica de
Estudiantes de Sistemas e Informática (SCESI), se presenta obvio el uso del
lenguaje PHP en su versión 5.3 o superior.

Al comienzo de la etapa de codificación se planteo el aprovechamiento de las
librerías provistas por el framework de desarrollo de aplicaciones Zend, pero
a causa del conjunto heterogéneo de personas que componen el grupo de PHP se
opto por implementar todas las características con funciones puras del
lenguaje PHP.

Se utilizará cualquier sistema operativo que posea una terminal tipo UNIX, en
este caso, cualquier distribución del sistema operativo GNU/Linux.

En esta primera etapa del proyecto no se hará uso intensivo del lenguaje
Javascript, solo se utilizara lo básico del lenguaje HTML, un poco de
javascript utilitario y un poco de CSS simple.

\section{Justificación}
Este proyecto posee intrínsecamente cualidades educativas, primeramente para
el grupo de PHP, esta el aprendizaje sobre la construcción de aplicaciones web
mas complejas, haciendo uso de las funciones que posee el lenguaje para el
diseño de arquitecturas orientadas a objetos.

También se encuentra inherentemente el aprendizaje avanzado de la linea de
comandos, para el mismo grupo de PHP, y la creación de una herramienta para la
introducción al estudio de sistemas operativos.

\begin{thebibliography}{99}

\bibitem{Stephenson} Stephenson, Neal (1999).\\
En el principio. . . fue la línea de comandos.\\

\end{thebibliography}

\end{document}
